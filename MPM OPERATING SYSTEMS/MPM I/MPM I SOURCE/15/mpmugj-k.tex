.MB +5
.MT -3
.LL 65
.PN 148
.HE MP/M User's Guide
.FT   (All Information Herein is Proprietary to Digital Research.)
.sp
.ce
APPENDIX J:  MP/M DDT Enhancements
.pp
The  following commands have been added to the MP/M  debugger
to  provide  a  function  similar to CP/M's  SAVE  command  and  to
simplify the task of patching and debugging PRL programs.
.sp 2
W:  WRITE DISK
.pp
The  purpose  of  the WRITE DISK command is  to  provide  the
capability to write a patched program to disk.   A single parameter
immediately  follows  the 'W' which is the number of  sectors  (128
bytes/sector)  to be written.   This parameter is entered in  hexadecimal.
.sp 2
V:  VALUE
.pp
The purpose of the VALUE command is to facilitate use  of the
WRITE DISK command by computing the parameter to follow the 'W'.  A
single parameter immediately follows the 'V' which is the NEXT
location following the last byte to be written to disk.
.pp
Normally a user would read in a file, edit it, and then write
it back to disk.  The read command produces a value for NEXT.  This
value  can be entered as a parameter following the 'V' command  and
the number of sectors to be written out using the 'W' command  will
be computed and displayed.
.sp 2
N:  NORMALIZE
.pp
The  purpose  of the NORMALIZE command is to relocate a  page
relocatable  file which has been read into memory by the  debugger.
To  debug  a  PRL  program the user would read it in  with  the 'R'
command  and  then use  the 'N' command to relocate it  within  the
memory segment the debugger is executing.
.sp 2
B:  BITMAP BIT SET/RESET
.pp
The purpose of the BITMAP BIT SET/RESET command is to  enable
the user to update the bitmap of a page relocatable file.   To edit
a  PRL  file the user would read the file in,  make changes to  the
code,  and  then determine the bytes which needed relocation  (E.G.
the  high  order  address bytes of  jump  instructions).   The  'B'
command  would then be used to update the bit map.   There are  two
parameters specified, the address to be modified (0100H is the base
of the program segment),  followed by a zero or a one.   A value of
one specifies bit setting.
.AD
.bp
.sp 2
.ce
APPENDIX K:  Page Relocatable (PRL) File Specification
.sp
.pp
Page  relocatable  files  are  stored  on  diskette  in  the
following format:
.li

Address:    Contents:
-------     --------

0001-0002H  Program size

0004-0005H  Minimum buffer requirements (additional memory)

0006-00FFH  Currently unused, reserved for future allocation


0100H + Program size  =  Start of bit map

.ad
.pp
The  bit map is a string of bits identifying which bytes  are
to be relocated.  There is one bit map byte per 8 bytes of program.
The  most  significant  bit (7) of the first byte of  the  bit  map
indicates  whether  or not the first byte of the program is  to  be
relocated.  A bit which is on indicates that relocation is
required.  The next bit,  bit(6),  of the first byte of the bit map
corresponds to the second byte of the program.
.br
