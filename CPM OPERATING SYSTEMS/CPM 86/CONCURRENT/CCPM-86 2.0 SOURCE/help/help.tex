.po 4
.mb 0
.mt 0
.pl 66
.ll 65
.op
.in 0
.nf
///1abort
Syntax:

ABORT programname 
ABORT programname n
.fi

ABORT immediately stops execution of the program specified 
by programname.  If you want to abort a program running on 
another console, include the number (n) of the console running 
the program. 

Use CTRL-C to abort a program running on the current virtual
console, and the ABORT command for programs running on another
virtual console.  If you abort more than one program, the ABORT 
commands are executed in the order given.
.nf
///2Examples:

In the following example, the ABORT command used to abort the 
program TYPE executing on console number 1 is executed from another
virtual console. The user number does not affect ABORT. 

.in 4
.nf
A>TYPE DOCUMENT.TXT

Dear Sir:
The company is pleased to inform you tha 

5B>ABORT type 1 <CR>
5B> 
.in 0
.fi

In the above example, assume that the TYPE command was issued
from virtual console 1.  The TYPE command is aborted from virtual
console 3.  
.nf
///1asm86
Syntax:

ASM86 filespec {$options}

Purpose:
.fi

ASM-86 assembles assembly language statements, producing a file
in hexadecimal format, a print file and a symbol table file.  The
assumed filetype of source and included files is A86.  The
special characters X, Y and Z indicate output to the console,
output to the printer, and zero output, respectively. If no format
is specified, Digital Research format (FD) is assumed. 
.nf

///2Examples
Examples:
.in 4

A>ASM86 PROG
A>ASM86 PROG $SZ PX
A>ASM86 PROG $HB PY AC

.in 0
///2Options
Syntax:

ASM86 filespec ($Ad Hd Pd Sd Fd)

A   source file drive - .A86
d = (logical drives A-D)

H   hex file drive    - .H86
d = (logical drives A-D, X,Y,Z)

P   print file drive  - .LST
d = (logical drives A-D, X,Y,Z)

S   symbol file drive - .SYM
d = (logical drives A-D, X,Y,Z)

F   format of hex file- .H86
d = (D=Digital Research, I=Intel)
.nf
///1buffered 
.fi
You can switch the current (foreground) virtual console into 
the background by selecting a different virtual console for display. 
If the switched-out virtual console is in Buffered Mode, any program 
output  to  that console is stored in a disk file. 
Then, when the 
background virtual console is again switched into the foreground, 
any output stored in the file is displayed on the monitor.
Use CTRL O to flush the disk buffer, that is, to skip over
the display of buffered output.

The other virtual console mode is Dynamic Mode.  Use the VCMODE 
command to switch consoles from one mode to the other.
.nf

///1commands
Concurrent CP/M-86 command line syntax:

A> <command> {command tail} <cr>
.fi

Concurrent CP/M-86 accepts the command lines you type following the
system prompt.  The command keyword identifies the system command 
or program to be executed. The optional command tail can consist of 
a filespec or various command parameters, depending upon the specific
command. To complete the command, press the RETURN key <cr>. File 
specifications used in command lines are composed of the following 
parts:

.nf
        {d:}filename{.typ}{;password}

        d:              is an optional drive specifier
        filename        is the 1- to 8-character file name
        .typ            is an optional file type
        ;password       is an optional 1- to 8-character password

///1date
Syntax:

DATE {DD/MM/YY HH:MM:SS | P}

Purpose:
.fi

The DATE utility allows you to set the system clock to the correct 
date and time. It also allows you to display the current date and 
time on the current virtual console. 

The date is represented in conventional day-month-year format, while 
the time is represented in 24-hour clock format: 00:00:00 to 11:59:59
indicates AM, and 12:00:00 to 23:59:59 indicates PM.

.nf
///2examples
Examples:

      A>DATE                ; displays the current date and time 
      Fri 02/18/83  14:22:23

      A>DATE 02/17/83  11:34:00 ; sets the current date and time

      Press any key to set time <cr>

      Th 02/17/83 11:34:00

      A>DATE P

The DATE P command causes the system to display the current
date and time continuously. Pressing any key cancels the display.

.nf
///1ddt86
Syntax:  

DDT86 {filespec}

Purpose:
.fi

DDT86 aids debugging of 8086 and 8088 programs.  DDT86 assumes
a default filetype of CMD.  If a file specification is not included, 
DDT86 is loaded into User Memory without a test program. The test 
program is then loaded using the E command. If the optional file 
specification is included in the command, both DDT86 and the test 
program file specified by filespec are loaded into memory. DDT86 
does not support passwords.  To exit DDT86, press CTRL-C.
.nf
///2Examples
Examples:
.in 4

A>DDT86
A>DDT86 PROGRAM1
A>DDT86 PROGRAM2.CMD
A>DDT86 B:PROGRAM3.CMD

.ti -4
DDT86 Commands:

-D
-L1008:0,4F
-SW23a
 
.nf
.in 0
///2commands
.fi
DDT86 Command Summary

The command character can be followed by one or more arguments.
Separate arguments from one another by commas or spaces; no spaces 
are allowed between the command character and the first argument. 
.nf

As              (Assemble)      Enter Assembly Language Statements
Bs,f,s1         (Block Compare) Compare Blocks of Memory
D(W)(s(,f))     (Display)       Display Memory in Hexadecimal and ASCII
Efilespec       (Execution)     Load Program for Execution
Fs,f,bc         (Fill)          Fill Memory Block with a byte constant
FWs,f,wc        (Fill Word)     Fill Memory Block with a word constant
G(s)(,b1(,b2))  (Go)            Begin Execution with optional breakpoints
Hwc1,wc2        (Hex)           Hexadecimal Sum and Difference
Icommand tail   (Input)         Set Up Input Command Line
L(s(,f))        (List)          List Memory in Mnemonic Form
Ms,f,d          (Move)          Move Memory Block
QI(W)n                          Read From I/O Port
QO(W)n,v                        Write To I/O Port
Rfilespec       (Read)          Read Disk File into Memory
S(W)s           (Set)           Set Memory Values
SRs,f,<string>                  Search For String
T(n)            (Trace)         Trace Program Execution
TS(n)           (Trace)         Trace and Show All Registers
U(n)            (Untrace)       Monitor execution without Trace
US(n)           (Untrace)       Monitor and Show all Registers
V               (Verify)        Show Memory Layout after Disk Read
Wfilespec       (Write)         Write Content of Block to Disk
X(r)            (Examine)       Examine and Modify CPU Registers
.in 0
.nf
///3parameters

DDT-86 Command Parameters

Parameter   Replace With

   bc       byte constant
   b1       breakpoint one
   b2       breakpoint two
   d        destination for data
   f        final address
   n        number of instructions
            to execute
   r        register or flag name
   s        starting address
   s1       second starting address
   W        word 16-bit
   wc       word constant
.in 0
.nf
///1dir
Syntax: 

DIR (filespec)

Purpose:
.fi

Displays the names of non-system (DIR) files in the directory
of an on-line diskette.  Use DIR [SYS] to find SYStem files.

.nf
///2Examples
Examples:
.in 4

A>DIR
A>DIR B:
A>DIR C:MYFILE.DAT
A>DIR *.CMD
A>DIR A*.A86
A>DIR PROG???.H86
A>DIR PROGRAM.*
A>DIR [SYS] B:UTILITY.CMD
.in 0
.nf
///1dynamic 
.fi
You can switch the current (foreground) virtual console into 
the background by selecting a different virtual console for display. 
If the switched-out virtual console is in Dynamic Mode, any program 
output to that console is stored in a reserved area of memory. 
Then, when the 
background virtual console is again switched into the foreground, 
any output stored in memory is displayed on the monitor.

However, the memory area that stores screen data has a capacity of 
only one screen. Previous program output can be replaced by later program
output if more than one screen 
of data is output while the virtual console is switched-out. 
Use the VCMODE command to switch a console from Dynamic to Buffered 
Mode if you anticipate a program outputting more than one screen of 
data to its virtual console while the console is switched-out.
.nf
///1ed
Syntax:

ED input-filespec {d:|output-filespec}

Purpose:
.fi

ED is the Concurrent CP/M-86 character file editor. Rename or
redirect the new version of the file by specifying the
destination drive or destination filespec. 

Chapter 4 of the Concurrent CP/M-86 User's Guide is devoted to
ED. 
.nf

///2Commands
.nf
Note:   CP points to the current character in the edit buffer.     

nA            append n lines from original file to memory buffer
0A            append file until buffer is one-half full
#A            append file until buffer is full (or end of file)
B, -B         move CP to the beginning (B) or bottom (-B) of buffer
nC, -nC       move CP n characters forward (C) or back (-C) through buffer
nD, -nD       delete n characters before (-D) or from (D) the CP
E             save new file and return to Concurrent CP/M-86
Fstring^Z     find character string
H             save new file, re-edit using new file as the original file
I<cr>         enter insert mode
Istring^Z     insert string at CP
Jsearch_str^Zins_str^Zdel_to_str     juxtapose strings
nK, -nK       delete (kill) n lines from the CP
nL, -nL       move CP n lines
nMcommand     execute command n times
n, -n         move CP n lines and display that line
n:            move to line n
:ncommand     execute command through line n
Nstring       extended find string
O             return to original file
nP, -nP       move CP n lines forward and display n lines at console
Q             abandon new file, return to Concurrent CP/M-86
R             read X$$$$$$$.LIB file into buffer
Rfilespec^Z   read filespec into buffer
Sdelete string^Zinsert string^Z     substitute string
nT, -nT       type n lines
U, -U         upper-case translation
V, -V         line numbering on/off
0V            display free buffer space
nW            write n lines to new file
0W            write until buffer is half empty
nX            write or append n lines to X$$$$$$$.LIB
nXfilespec^Z  write n lines to filespec or append if previous X
              command applied to the same file
0x            delete file X$$$$$$$.LIB
0xfilespec^Z  delete filespec
nZ            wait n seconds

.nf
///2Examples
Examples:
.in 4

A>ED TEST.DAT
A>ED TEST.DAT B:
A>ED TEST.DAT TEST2.DAT
A>ED TEST.DAT B:TEST2.DAT

.ti -4
Commands:

:#a
:p
:4c0tt
:e

.in 0

.nf
///1era
Syntax: 

ERA filespec

Purpose:
.fi

ERA erases a file or group of files.  ERA accepts ambiguous
file specifications.  The ERAQ command works like ERA, but prompts
you before performing the erasure.

.nf
///2Examples
Examples:
.in 4

A>ERA DRAFT.BAK
A>ERA B:LETTER.DAT
A>ERA C:LETTER.*
A>ERA D:*.BAK
A>ERA B:*.*
.in 0

.nf
///1eraq
Syntax: 

ERAQ filespec

Purpose:
.fi

ERAQ erases a file or group of files, prompting you before each
deletion.  ERAQ accepts ambiguous filenames. 

The ERA command works as ERAQ does, but does not prompt you
before performing the erasure. 

.nf
///2Example
.in 4

1A>ERAQ B:*.CMD

B:ASM86      CMD ?y
B:FUNCTION   CMD ?n
B:SDIR       CMD ?y
B:DSKRESET   CMD ?y
1A>

.in 0
.fi
In the example, the user instructs ERAQ to delete all the files
except C:FUNCTION.CMD. 

.nf
///1filename
.fi
Concurrent CP/M-86 identifies every file by its unique file
specification.  The term "filespec" is an abbreviation for file 
specification. A filespec can have four parts:  
.nf

             d:  filename  .typ  ;password

.fi
.in +10
.ti -10
 d:        represents the optional drive specification, which can
range from A through P, followed by a colon. 

.ti -10
 filename  represents the required filename, which can be
1 to 8 alphabetic or numeric characters. 

.ti -10
 .typ       represents the optional filetype, which can be 0 to 3
alphabetic or numeric characters preceded by a period. 

.ti -10
 password   represents the optional password, which can be 1 to 8
alphabetic or numeric characters.
.in 0
.fi

Valid combinations of the elements of a file specification are
shown below: 

.nf
.in 4
o  d:
o  filename
o  d:filename
o  filename.typ
o  d:filename.typ
o  filename.typ;password
o  d:filename.typ;password
.in 0
.fi

Certain Concurrent CP/M-86 commands select and process several
files if "wildcard" characters appear in the filename or
filetype. The two wildcard characters are ?, which matches an
single letter in the same position, and *, which matches any
character at that position and any other characters remaining in
the filename or filetype.
.nf
///2conventions
.fi
Command Summary Conventions

The command summary lists each CP/M-86 command in alphabetical
order using the following special symbols to define command
syntax: 

.nf
.in 4
[]   indicate an optional item
d    indicates a drive number
n    indicates a number
|    separates choices
<cr> indicates a carriage return
^    indicates the Control (CTRL) key
RW   means Read Write 
RO   means Read Only
SYS  means System attribute
DIR  means Directory attribute
.in 0
.nf

///1gencmd
Syntax:

GENCMD filespec {options}

Purpose:
.fi

Converts hexadecimal object file (filetype assumed to be .H86)
into executable file (of type .CMD). Switches controlling the
type of memory model and group addresses can be included.  
All values are hexadecimal and are paragraph values. 

.nf
///2Examples
Examples:
.in 4

A>GENCMD PROG
A>GENCMD PROG1 8080
A>GENCMD PROG2 DATA[M20]
A>GENCMD PROG3 DATA[B4C,M260,XFFF]
A>GENCMD PROG4 CODE[A40] DATA[M30]

.in 0
///2options
Syntax:

GENCMD filespec  {8080
                  CODE[An,Bn,Mn,Xn]
                  DATA[An,Bn,Mn,Xn]
                 STACK[An,Bn,Mn,Xn]
                 EXTRA[An,Bn,Mn,Xn]}

   A = Absolute memory location
   B = Beginning address of group in .H86 file
   M = Minimum memory required
   X = Maximum memory required

All values represented by n are hexadecimal paragraph addresses.
.nf
///1help
Syntax:
 
HELP {topic,subtopic1, ... ,subtopic8}

Purpose:
.fi

Supplies information on Concurrent CPM-86 commands.  HELP, when 
followed by a topic and an optional sequence of subtopics, displays
information about that topic on your screen. At the HELP> prompt, 
you can enter a topic and optional subtopics. Pressing <cr>
without a topic specified terminates HELP. 
.nf
///2examples

Examples:
.in 4

A>HELP
A>HELP dir
A>HELP dir options
HELP> help
HELP> set examples
.in 0
.nf
///1initdir
Syntax:

INITDIR d:

Purpose:
.fi

INITDIR initializes a disk directory to allow time and date stamping
on that disk. Attempting to enable time and date stamping on a directory
that has not been reformatted by INITDIR will result in an error message.
INITDIR only works on disks that have already been formatted according
to your computer manufacturer's instructions. 
.nf
///2example
Example:

      A>INITDIR C:

      INITDIR WILL ACTIVATE TIME STAMPS FOR SPECIFIED DRIVE
      Do you really want to re-format the directory?  C (Y/N)?
.fi

Answer with a "Y" to continue. If the specified disk was already formatted
for time and date stamps, INITDIR displays the message:

.nf
      Directory already re-formatted
      Do you want to continue (Y/N)?
.fi 

If you answer "Y" to this question, INITDIR asks:

.nf
      Do you want the existing time and date stamps cleared (Y/N)?
///1pip
Syntax:

PIP filespec{[Gn]}=filespec{[option-list]}{,filespec[option-list],...}
PIP filespec{[Gn]}|dev=filespec{[option-list]}|dev{[option-list]}{,...}

Purpose:
.fi

Copies, combines and transfers files between peripheral devices. 
The first filespec is the destination. The second filespec is the 
source.  Alternately, the source or destination can be any Concurrent 
CP/M-86 logical device.  You can specify multiple source filespecs, 
with options, to concatenate several files into one. An option-list 
is any combination of the available options. [Gn] is the only 
option allowed with the destination filespec.
.nf
///2Examples
A>PIP B:=A:DRAFT.TXT                 ; Copy from one disk to another
A>PIP B:NEWDRAFT.TXT=A:OLDDRAFT.TXT  ; Copy a file and rename it
A>PIP <cr>                           ; Load PIP for multiple commands
A>PIP B:=C:*.*                       ; Copy multiple files
A>PIP B:=*.TXT [AV]                  ; Archive and verify options
A>PIP B:NEW.DAT=FILE1.DAT,FILE2.DAT  ; Combine multiple files
A>PIP NEWDRAFT.TXT[G1]=OLDDRAFT.TXT  ; Copy, rename and place in user 1
A>PIP NEWDRAFT.TXT=OLDDRAFT.TXT[G1]  ; Copy, rename and get from user 1
A>PIP B:FUNFILE.SUE=CON:             ; Copy to file from console
A>PIP LST:=CON:                      ; Copy to printer from console
A>PIP LST:=B:DRAFT.TXT[T8]           ; Expand tabs option
A>PIP PRN:=B:DRAFT.TXT               ; Copy file to printer, expand tabs,
                                     ; insert form-feeds every page
.nf
///2options
A    - Archive option.
Dn   - Delete any characters past column n.
E    - Echo transfer to console.
F    - Filter form-feeds from source data.
Gn   - Get from or go to user code n. (default n=current user num.)
H    - Test for valid Hex format.
I    - Ignore :00 Hex data records and test for valid Hex format.
K    - Kill display of filenames on console.
L    - Translate upper case to lower case.
N    - Number output lines
O    - Object file transfer, ^Z ignored.
Pn   - Set page length to n. (default n=60)
Qs^Z - Quit copying from source at string s.
R    - Read files that have been set to SYStem.
Ss^Z - Start copying from the source at the string s.
Tn   - Expand tabs to n spaces.
U    - Translate lower case to upper case.
V    - Verify that data has been written correctly.
W    - Write over Read Only file without console query.
Z    - Zero the parity bit.
.nf
///1printer
Syntax:

PRINTER {n}

Purpose:

.fi
PRINTER displays or selects the printer device attached to the
current virtual console. Several consoles can share the same 
printer, but only one process can use a given printer at a time. 
When you enter the PRINTER command without a number, the system 
returns the number of the printer assigned to the current virtual 
console. To select a printer, enter the PRINTER command followed 
by the printer number to be attached. Note: Refer to the instruction 
manual supplied by your hardware manufacturer to determine how many 
printers your system supports.
.nf
///2examples
Examples:

       A>PRINTER            ; displays the current printer number
       Printer Number = 1

       A>PRINTER 3          ; sets the current printer number
       Printer Number = 3
.nf
///1ren
Syntax:

REN {d:}newname{.typ}=oldname{.typ}

Purpose:

.fi
REN changes the name of the existing file (specified by oldname) to 
a new name (specified by newname). You cannot specify two different
drives. If the file given by newname is already present in the
directory, REN displays the following message on the screen:
.in 4

.nf
Not renamed: Newfile already exists, delete (Y/N)?
.in 0
.nf

///2Examples
Examples:
.in 4

A>REN NEWFILE.DAT=OLDFILE.DAT
A>REN B:NEWFILE.DAT=OLDFILE.DAT
A>REN B:NEWLIST=OLDLIST
A>REN NEWFILE.DAT=OLDFILE.DAT

Not renamed: Newfile already exists, delete (Y/N)? Y

NEWFILE.DAT=OLDFILE.DAT

.in 0
.fi
In the last example, the new filename already exists. The user
overrides the error by entering a Y at the prompt. 
.nf
///1sdir
.nf
Syntax:

SDIR {d:}{filespec}{,filespec}{[option]|[option=modifier]}
.fi

The SDIR utility is an enhanced version of the DIR utility.
SDIR can search for files on any or all drives, in any or all 
user areas. Only one option list is allowed. The most efficient 
way to become familiar with SDIR is to use it. SDIR does not 
change any information on diskette or in memory, so you can 
experiment with it freely. 
.nf
///2Examples 
A>SDIR [xfcb] D:*.CMD
A>SDIR [short,ro] A: B: C:
A>SDIR [user=3,exclude] *.CMD 
A>SDIR [size,rw] D:
A>SDIR [user=all,drive=all,sys] *.PLI *.CMD *.A86
///2options
[ATTRIBUTES]  Displays if file attributes F1-F4 are set.
[DRIVE=d:]    Displays files on the specified drive only.
[EXCLUDE]     Displays only files that do not match filespec.
[FF]          Prints a Form Feed character at the start of each header.
[FULL]        The default SDIR option. 
[LENGTH=n]    Displays a new (Page) header every n lines.
[RO]          Displays Read-Only files only.
[RW]          Displays Read/Write attribute files only.
[SIZE]        Displays only file name and size.
[SYS]         Displays SYStem attribute files only.
[USER=n]      Displays files on user n only.
[USER=all]    Displays files on all user numbers.
[XFCB]        Displays only files with XFCBs or date stamping.
///1set
.nf
Syntax:

SET d:|filespec [option{=modifier}]{,d:|filespec [option{=modifier}],...}

.fi
Use SET to control password protection and time stamping of files, 
and to set file and drive attributes. The SET command always requires 
a parameter. SET options are always enclosed in square brackets.  

SET options affect drives and files.  Separate multiple options 
and commands by commas. 
.nf
///2Examples
A>SET [HELP]     
A>SET [NAME=mylabel.dsk]
A>SET *.CMD [SYS,RO,PASSWORD=secret,PROTECTION=READ]
A>SET *.HEX [RW,PROTECTION=NONE,DIR]
A>SET *.TEX [PASSWORD=secret,PROTECTION=WRITE]
A>SET ONE.TEX,TWO.TEX [PROTECTION=NONE],*.PRN [SYS]
A>SET [DEFAULT=secret]
A>SET [CREATE=ON]
A>SET [ACCESS=ON]
A>SET [UPDATE=ON]
A>SET B:[RO]
A>SET B:[RW]
///2options
[ACCESS=ON|OFF]       Turn access time stamps on/off.
[CREATE=ON|OFF]       Turn creation time stamps on/off.
[DEFAULT=password]    Specify a default password.
[DIR]                 Set file DIRectory attribute.
[F1|F2|F3|F4=ON|OFF]  Set file attribute bit (n = 1 to 4)
[HELP]                Display a list of examples.
[MAKE=ON|OFF]         Turn automatic creation of XFCBs on/off.
[NAME=lablname.typ]   Specify directory label name.
[PASSWORD=password]   Specify file or directory label password.
[PROTECTION=READ|WRITE|EDIT|NONE]   Set level of password protection.
[RO]                  Set file or drive to Read-Only.
[RW]                  Set file or drive to Read/Write.
[SYS]                 Set file SYStem attribute.
[UPDATE=ON|OFF]       Turn update time stamps on/off.
///1show
.nf
Syntax:

SHOW {d:}{option}

.fi
SHOW by itself displays the drive, the Read Only or Read Write
mode for the optionally specified drive, and the remaining space 
in kilobytes for all logged-in drives in the system. 

Use the SHOW options to display drive characteristics, active 
user numbers, or the directory label. SHOW HELP displays a
list of available options.
.nf
///2Examples
SHOW 
SHOW SPACE
SHOW DRIVES
SHOW USERS
SHOW LABEL
SHOW HELP
SHOW A:SPACE
SHOW B:DRIVE
SHOW C:USERS
SHOW D:LABEL
///2options
SHOW SPACE    ; Same as the SHOW display. 

SHOW DRIVES   ; Displays the drive characteristics of logged-in
              ; drives on the system, or for a specified drive. 

SHOW USERS    ; Displays the current user number and all user areas 
              ; on the drive that have files assigned to them.

SHOW LABEL    ; Returns a display of the optional directory label, 
              ; if it has been created.  

SHOW HELP     ; Displays a list of the SHOW options.  

SHOW d:       ; SHOW with the optional drive specifier displays 
              ; information for the specified drive only. 
///1submit
Syntax:

SUBMIT filespec (actual parameters)

Purpose:
.fi

SUBMIT processes a command file with a filetype of .SUB
consisting of of CP/M-86 commands given one command per line. Any
optional parameters (such as a drive or filespec) following the
filespec in the command line are substituted for their
corresponding formal parameters ($1,$2,$3...) in the SUBMIT file.

.nf
///2Examples
.in 4

.nf
A>SUBMIT START
A>SUBMIT B:START
A>SUBMIT START C: LETTER
.nf
.in 0

///1systat
Syntax:

SYSTAT {[option {C} {n}]

Purpose:
.fi

The SYSTAT utility shows the internal state of Concurrent CP/M-86.
It is useful for program and system development. SYSTAT displays 
memory allocation, current processes, system queue activity, and
many parameters associated with system data structures. SYSTAT can
present either a static picture or a continuous, real-time display
of these system parameters.
.nf

///2examples
Examples:

      A>SYSTAT <cr>
.fi 

This command invokes the menu-driven feature of SYSTAT. The utility
responds by displaying the following menu:
.nf

.in 6
Which Option ?

.in +3
H(elp)
M(emory)
O(verview)
P(rocesses - All)
Q(ueues)
U(ser Processes)
E(xit)

.in -3
->_
.in 0
.fi


Typing the appropriate letter in response to the menu obtains the
associated display. To use SYSTAT without the menu, specify the menu
option letter in the command, like this:

.nf
.in 6
A>SYSTAT [O] <cr>    ; this command displays a snapshot 
                     ; overview of the system.
A>SYSTAT [MC] <cr>   ; this command displays a continuous
                     ; picture of memory allocation.
A>SYSTAT [UC10] <cr> ; this command displays snapshots
                     ; of the user processes every 10 seconds.
.in 0

///2options
.fi
.in +8
.ti -8
[Cn]    When specified in the SYSTAT command line along with another 
option, displays that option continuously, updating the display 
in real-time,  until a key is pressed.  Following the C 
option with a two-digit number, n, causes SYSTAT to update 
the display every n seconds.
.sp
.ti -8
[E]     Returns you to the system prompt level from the menu.
.sp
.ti -8
[M]     Displays all memory potentially available to users, but 
does not display restricted memory.  The partitions are 
listed in memory-address order.
.sp
.ti -8
[O]     Displays an overview of the system generation parameters.
.sp
.ti -8
[P]     Displays all system processes and resources they use.
.sp
.ti -8
[Q]     Displays all queues and their readers, writers and owners.
.sp
.ti -8
[U]     Displays only user-initiated processes (similar to [P]).
.nf
.in 0
///1type
Syntax: 

TYPE filespec

Purpose:
.fi

TYPE displays contents of an ASCII file on the screen.  Press any
key to discontinue the display.  TYPE does not accept wildcard
filespecs. Entering a ^P prior to the type command causes the
output to be echoed to the printer until another ^P is entered. 

.nf
///2Examples
Examples:
.in 4

.nf
A>TYPE letter.dat
B>TYPE a:document.law
C>TYPE program.bas
D>TYPE program.a86
.fi
.in 0

.nf
///1user
Syntax: 

USER (number 0 - 15)

Purpose:
.fi

USER displays and changes the current user number.  USER with no
command tail displays the current user number.  USER with a
number from 0 to 15 changes the current user number to the number
specified by n.  CP/M assumes a default user number of 0.  Files
set to SYStem on USER 0 are available to all USER areas as Read-
Only. 
.nf

///2Examples
Examples:
.in 4

.nf
A>USER
B>USER 2
A>USER 7
.fi
.in 0

///1vcmode
.nf
Syntax:

VCMODE {option}
 
.fi
VCMODE specifies background operating modes for the four virtual
consoles. When a virtual console is switched-out, it operates in either the
Dynamic Mode or the Buffered Mode. See the HELP file explanations under
BUFFERED  and DYNAMIC  for more information.
.nf
///2Examples

A>VCMODE
A>VCMODE dynamic
A>VCMODE buffered 
A>VCMODE size=5

///2options

DYNAMIC    Switches the console to Dynamic mode. If a console is in 
           Dynamic Mode and you switch it out, data normally output 
           to the screen fills a space in memory  reserved for such 
           data. The oldest data is lost as the new data is written 
           in, therefore when you switch back to this console, some 
           data may be lost. 

BUFFERED   If a console is in Buffered Mode and switched out,  data 
           output from a running program goes into a buffer file on 
           diskette. Thus no data is lost.

SIZE=n     Specifies the maximum size of the buffer file in  Kbytes 
           and switches the console to Buffered mode.
