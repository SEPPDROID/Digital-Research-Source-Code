.he
.bp odd
.mt 5
.mb 6
.pl 66
.ll 65
.po 10
.hm 2
.fm 2
.ft  All Information Presented Here is Proprietary to Digital Research
.sp 3
.ce 2
.sh
Section 6
.sp
.sh
Installing and Adapting the Distributed BIOS and CP/M-8000
.tc 6  Installing and Adapting the Distributed BIOS and CP/M-8000
.sp 2
============ THIS ENTIRE SECTION NEEDS TO BE RE-WRITTEN ============
============ FIRST WE NEED TO KNOW HOW CP/M IS DISTRIBUTED =========
.sh
6.1  Overview
.he CP/M-8000 system Guide                               6.1  Overview
.tc    6.1  Overview
.ix CP/M-8000, installing
.pp 5
The process of bringing up your first running CP/M-8000 system is
either trivial or involved, depending on your hardware environment.
Digital Research supplies CP/M-8000 in a form suitable for booting
on a Zilog EXORmacs development system.  If you have an EXORmacs,
you can read Section 6.1 which tells how to load the distributed system.
Similarly, you can buy or lease some other machine which already runs
CP/M-8000.
.pp
If you do not have an EXORmacs, you can use the S-record files supplied
with your distribution disks to bring up your first CP/M-8000 system.  
This process is discussed in Section 6.2.  
.ix EXORmacs
.ix S-record files
.sp 2
.sh
6.2  Booting on an EXORmacs
.he CP/M-8000 System Guide                 6.2  Booting on an EXORmacs
.tc    6.2  Booting on an EXORmacs
.ix boot disk
.pp 5
The CP/M-8000 disk set distributed by Digital Research includes disks
boot and run CP/M-8000 on the Zilog EXORmacs.  You can use the distribution 
system boot disk without modification if you have a Zilog EXORmacs system 
and the following configuration:
.ix configuration requirements
.sp 2
.in 8
.ti -3
1) 128K memory (minimum)
.sp
.ti -3
2) a Universal Disk Controller (UDC) or Floppy Disk Controller (FDC)
.sp
.ti -3
3) a single-density, IBM 3740 compatible floppy disk drive 
.sp
.ti -3
4) an EXORterm
.in 0
.sp 2
To load CP/M-8000, do the following:
.ix CP/M-8000, loading
.sp 2
.in 8
.ti -3
1) Place the disk in the first floppy drive (#FD04 with the UDC or #FD00
with the FDC).
.ix MACSbug
.ix UDC
.ix FDC
.sp
.ti -3
2) Press SYSTEM RESET (front panel) and RETURN (this brings in MACSbug).
.sp
.ti -3
3) Type "BO 4" if you are using the UDC, "BO 0" if you are using the
FDC, and RETURN.  CP/M-8000 boots and begins running.
.sp 2
.in 0
.sh
6.3  Bringing Up CP/M-8000 Using the S-record Files
.he CP/M-8000 System Guide           6.3  CP/M-8000 with S-record Files
.tc    6.3  Bringing up CP/M-8000 Using S-record Files
.pp
The CP/M-8000 distribution disks contain two copies of the CP/M-8000 operating
system in Zilog S-record form, for use in getting your first CP/M-8000
system running.  S-records (described in detail in Appendix F) are a simple 
ASCII representation for absolute programs.  The two S-record 
systems contain the CCP and BDOS, but no BIOS.  One of the S-record systems
resides at locations 400H and up, the other is configured to occupy the top
of a 128K memory space.  (The exact bounds of the S-record systems may vary
from release to release.  There will be release notes and/or a file named 
README describing the exact characteristics of the S-record systems 
distributed on your disks.)  To bring up CP/M-8000 using the S-record files, 
you need:
.ix S-records, bringing up CP/M-8000
.ix S-record systems
.ix CCP
.IX BDOS 
.ix README file
.sp 2
.in 8
.ti -3
1) some method of down-loading absolute data into your target system
.ix absolute data, down-loading
.sp
.ti -3
2) a computer capable of reading the distribution disks (a CP/M-based
computer that supports standard CP/M 8-inch diskettes)
.sp
.ti -3
3) a BIOS for your target computer
.sp
.in 0 
.pp
Given the above items, you can use the following procedure to bring a working
version of CP/M-8000 into your target system:
.in 8
.ix _init entry point
.sp
.ti -3
1) You must patch one location in the S-record system to link it to your BIOS's
_init entry point.  This location will be specified in release notes and/or
in a README file on your distribution disks.  The patch simply consists of 
inserting the address of the _init entry in your BIOS at one long word
location in the S-record system.  This patching can be done either before
or after down-loading the system, whichever is more convenient. 
.ix S-record, long word location 
.ix _ccp entry point
.sp
.ti -3
2) Your BIOS needs the address of the _ccp entry point in the S-record 
system.  This can be obtained from the release notes and/or the README file.
.sp
.ti -3
3) Down-load the S-record system into the memory of your target computer.
.mb 4
.fm 1
.sp
.ti -3
4) Down-load your BIOS into the memory of your target computer.
.sp
.ti -3
5) Begin executing instructions at the first location of the down-loaded
S-record system.
.sp
.in 0
.pp
Now that you have a working version of CP/M-8000, you can use the tools 
provided with the distribution system for further development.
.sp
.ce
End of Section 6
.nx seven




