.bp
.cs 5
.mt 5
.mb 6
.pl 66
.ll 65
.po 10
.hm 2
.fm 2
.he CP/M-8000 Programmer's Guide              E.2  AS68 Error Messages
.ft All Information Presented Here is Proprietary to Digital Research
.tc         E.2.3  AS68 Internal Logic Error Messages
.sh
E.2.3  AS68 Internal Logic Error Messages
.pp 5
This section lists messages indicating fatal errors in the 
internal logic of AS68.  
If you receive one of these messages, contact the place you 
purchased your system for assistance.  You should provide the 
information below.
.mb 6
.fm 2
.sp
.in 8
.ti -3
1) Indicate which version of the operating system you are using.
.sp
.ti -3
2) Describe your system's hardware configuration.
.sp
.ti -3
3) Provide sufficient information to reproduce the error.  Indicate 
which program was running at the time the error occurred.  If possible, 
you should also provide a disk with a copy of the program.
.in 0
.sp 2
Errors:
.sp 
.nf
.in 5
.ti -2
o &  doitrd: buffer botch pitix=nnn  itbuf=nnn  end=nnn  
.ti -2
.sp
o &  doitwr: it buffer botch
.ti -2
.sp
o &  invalid radix in oconst
.ti -2
.sp
o &  i.t. overflow
.ti -2
.sp
o &  it sync error itty=nnn
.ti -2
.sp
o &  seek error on it file
.ti -2
.sp
o &  outword: bad rlflg
.fi
.in 0
.sp 2
.tc    E.3  BDOS Error Messages
.he CP/M-8000 Programmer's Guide              E.3  BDOS Error Messages
.sh
E.3  BDOS Error Messages
.pp 5
The CP/M-8000 Basic Disk Operating System, BDOS, returns fatal 
error messages at the console.  The BDOS error messages are 
listed below in alphabetic order with explanations and suggested 
user responses.
.bp
.ce
.sh
Table E-4.  BDOS Error Messages
.nf
.ll 60
.in 5
.sp
Message        Meaning
.fi
.sp
.in 20
.ti -15
CP/M Disk change error on drive x
.sp
The disk in the drive indicated by the variable "x" is 
not the same disk the system logged in previously.  When the 
disk was replaced you did not enter a CTRL-C to log in the 
current disk.  Therefore, when you attempted
to write to, erase, or rename a file on the current disk, the BDOS set the
drive status to read-only and warm booted the system.
The current disk in the drive was not overwritten.  The drive status 
was returned to read-write when the system was warm booted.  Each time a 
disk is changed, you must type a CTRL-C to log in
the new disk.  
.sp 2
.ti -15
CP/M Disk file error:  filename is Read-Only.
.ti -15
.sp 0
.nf
Do you want to:  Change it to read/write (C), 
.ti -15
.sp 0
or Abort (A)?
.fi
.sp
You attempted to write to, erase, or rename a file 
whose status is Read-Only.  Specify one of the options enclosed in parentheses.
If you specify the C option, the BDOS changes the status of 
the file to read-write and continues the operation.  The 
read-only protection previously assigned to the file is lost.
.pp
If you specify the A option or a CTRL-C, the program terminates 
and CP\M-8000 returns the system prompt.
.sp 2
.ti -15
CP/M Disk read error on drive x
.sp 0
.nf
.ti -15
Do you want to:  Abort (A), Retry (R), or Continue 
.ti -15
with bad data (C)?
.fi
.sp
.in 20
BDOS.  This message indicates a hardware error.  Specify one of the options 
enclosed in parentheses.  Each option is described below.
.in 0
.ll 65
.bp
.ce
.sh
Table E-4.  (continued)
.nf
.ll 60
.in 5
.sp
Message        Meaning
.fi
.sp
.in 20
.nf
Option        Action
.fi
.sp
.in 34  
.ti -14
A or CTRL-C   Terminates the operation and CP/M-8000 returns the 
system prompt.
.sp
.ti -14
R             Retries the operation.  If the retry fails, the 
system reprompts with the option message.
.sp
.ti -14
C             Ignores the error and continues program execution. 
Be careful if you use this option.  Program execution should 
not be continued for some types of programs.  For example, if 
you are updating a data base and receive this error but continue 
program execution, you can corrupt the index fields and the entire 
data base.  For other programs, continuing program execution is 
recommended.  For example, when you transfer a long text file and 
receive an error because one sector is bad, you can continue 
transferring the file.  After the file is transferred, review 
the file, and add the data that was not transferred due to the 
bad sector.
.sp 2
.in 20
.ti -15
CP/M Disk write error on drive x
.sp 0
.ti -15
.nf
Do you want to:  Abort (A), Retry (R), or 
.ti -15
Continue with bad data (C)?
.fi
.sp
BDOS.  This message indicates a hardware error.  Specify one of the options 
enclosed in parentheses.  Each option is described below.
.in 0
.ll 65
.bp
.ce
.sh
Table E-4.  (continued)
.nf
.ll 60
.in 5
.sp
Message        Meaning
.fi
.sp
.in 20
.nf
Option        Action
.fi
.sp
.in 34
.ti -14
A or CTRL-C   Terminates the operation and CP/M-8000 returns the 
system prompt.
.sp
.ti -14
R             Retries the operation.  If the retry fails, the 
system reprompts with the option message.
.sp
.ti -14
C             Ignores the error and continues program execution. 
Be careful if you use this option.  Program execution should 
not be continued for some types of programs.  For example, if 
you are updating a data base and receive this error but continue 
program execution, you can corrupt the index fields and the entire 
data base.  For other programs, continuing program execution is 
recommended.  For example, when you transfer a long text file and 
receive an error because one sector is bad, you can continue 
transferring the file.  After the file is transferred, review 
the file, and add the data that was not transferred due to the 
bad sector.
.sp 2
.in 20
.ti -15
CP/M Disk select error on drive x
.sp 0
.ti -15
Do you want to:   Abort (A), Retry (R)
.sp
There is no disk in the drive or the disk is not 
inserted correctly.  Ensure that the disk is
securely inserted in the drive.  If you enter the R option, the system 
retries the operation.  If you enter the A option or CTRL-C
the program terminates and CP\M-8000 returns the system prompt.
.in 0
.ll 65
.bp
.ce
.sh
Table E-4.  (continued)
.nf
.ll 60
.in 5
.sp
Message        Meaning
.fi
.sp
.in 20
.ti -15
CP/M Disk select error on drive x
.sp
The disk selected in the command line is outside the 
range A through P.  CP/M-8000 can support up to 16 drives, lettered A through 
P.  Check
the documentation provided by the manufacturer to find out which drives your
particular system configuration supports.  Specify the
correct drive code and reenter the command line.
.sp 2
.in 0
.ll 65
.tc    E.4  BIOS Error Messages
.he CP/M-8000 Programmer's Guide              E.4  BIOS Error Messages
.sh
E.4  BIOS Error Messages
.pp 5
The CP/M-8000 BIOS error messages are listed below in alphabetic order
with explanations and
suggested user responses.
.ix BIOS error messages
.ix error messages, BIOS
.sp 2
.ce
.sh
Table E-5.  BIOS Error Messages
.nf
.ll 60
.sp
.in 5
Message        Meaning
.fi
.sp
.in 20
.ti -15
BIOS ERROR -- DISK X NOT SUPPORTED
.sp
The disk drive indicated by the variable "X" is not supported by
the BIOS.  The BDOS supports a maximum of 16 drives, lettered A through
P.  Check the manufacturer's documentation for your system configuration to 
find out which of the BDOS drives your BIOS implements.  Specify the
correct drive code and reenter the command line.
.sp 2
.ti -15
BIOS ERROR -- Invalid Disk Status
.sp
The disk controller returned unexpected or 
incomprehensible information to the BIOS.  Retry the operation.  If the error 
persists, 
check the hardware.  If the error
does not come from the hardware, it is caused by an error in the internal 
logic of the BIOS.  Contact the place you purchased your 
system for assistance.  
You should provide the information below.
.in 0
.ll 65
.bp
.ce
.sh
Table E-5.  (continued)
.nf
.ll 60
.sp
.in 5
Message        Meaning
.fi
.sp
.in 23
.ti -3
1) Indicate which version of the operating system you are using.
.sp
.ti -3
2) Describe your system's hardware configuration.
.sp
.ti -3
3) Provide sufficient information to reproduce the error.  Indicate 
which program was running at the time the error occurred.  If possible, 
you should also provide a disk with a copy of the program.
.in 0
.ll 65
.sp 2
.tc    E.5  CCP Error Messages
.he CP/M-8000 Programmer's Guide               E.5  CCP Error Messages
.sh
E.5  CCP Error Messages
.pp 5
The CP/M-8000 Console Command Processor, CCP, returns two types of 
error messages at the console:  diagnostic and internal logic 
error messages.  
.sp 2
.tc         E.5.1  Diagnostic Error Messages
.sh
E.5.1  Diagnostic Error Messages
.pp
The CCP error messages are listed below in alphabetic order
with explanations and suggested user responses.
.sp 2
.ce
.sh
Table E-6.  CCP Diagnostic Error Messages
.nf
.ll 60
.sp
.in 5
Message        Meaning
.fi
.sp
.in 20
.ti -15
bad relocation information bits
.sp
This message is a result of a BDOS Program Load Function 
(59) error.  It indicates that the file specified in the command line is 
not a valid 
executable command file, or that the file has been corrupted.  Ensure that 
the file is a command file.  Section 3 of this manual
describes the format of a command file.  If the file has been 
corrupted, reassemble, or recompile the source file, and relink the file before you reenter 
the command line.
.in 0
.ll 65
.bp
.ce
.sh
Table E-6.  (continued)
.nf
.ll 60
.sp
.in 5
Message        Meaning
.fi
.sp
.in 20
.ti -15
File already exists
.sp
This error occurs during a REN command.  The name specified in the
command line as the new filename already exists.  Use the ERA command to 
delete the existing file if you 
wish to replace it with the new file.
If not, select another filename and reenter the REN command line.
.sp 2
.ti -15
insufficient memory or bad file header
.sp
This error could result from one of three causes:  
.sp
.in 23
.ti -3
1) The file is not a valid executable command file.  Ensure that you are 
requesting the correct file.  This error can occur
when you enter the filename before you enter the command for a 
utility.  Check the appropriate section of this manual or the 
CP/M-8000 Operating System User's Guide for the correct
command syntax before you reenter the command line.  If you are trying to run
a program when this error occurs, the program file may
have been corrupted.  Reassemble or recompile the source file 
and relink the file before you reenter the command line.  
.sp
.ti -3
2) The program is too large for the available memory.  Add more memory boards 
to the system configuration, or rewrite the 
program to use less memory.  
.sp
.ti -3
3) The program is linked to an absolute location in memory that cannot be 
used.  The program must be made relocatable, or linked to a usable 
memory location.  The BDOS Get/Set TPA Limits Function (63) returns 
the high and low boundaries of the memory space that is 
available for loading programs.
.in 0
.ll 65
.bp
.ce
.sh
Table E-6.  (continued)
.nf
.ll 60
.sp
.in 5
Message        Meaning
.fi
.sp
.in 20
.ti -15
No file
.sp
The filename specified in the command line does not exist.  Ensure that you 
use the correct filename and reenter the 
command line.
.sp 2
.ti -15
No wildcard filenames
.sp
The command specified in the command line does not accept 
wildcards in file specifications.  Retype the command line using a specific 
filename.
.sp 2
.ti -15
read error on program load
.sp
This message indicates a premature end-of-file.  The file is 
smaller than the header information indicates.  Either the file header has
been corrupted or the file was only partially written.  Reassemble, or 
recompile the source file, and relink the file before you reenter the
command line.
.sp 2
.ti -15
SUB file not found
.sp
The file requested either does not exist, or does not have a
filetype of SUB.  Ensure that you are requesting the correct file.  
Refer to the section on SUBMIT in the \c
.ul
CP/M-8000 Operating System User's Guide \c
.qu
for information on creating and using submit files.  
.sp 2
.ti -15
Syntax:  REN newfile=oldfile
.sp
The syntax of the REN command line is incorrect.  The correct
syntax is given in the error message.  Enter the REN command followed by a 
space, then the new filename,
followed immediately by an equals 
sign (=) and the name of the file you want to rename.
.in 0
.ll 65
.bp
.ce
.sh
Table E-6.  (continued)
.nf
.ll 60
.sp
.in 5
Message        Meaning
.fi
.sp
.in 20
.ti -15
Too many arguments:  argument?
.sp
The command line contains too many arguments.
The extraneous arguments are indicated by the variable "argument."  Refer to 
the \c
.ul
CP/M-8000 Operating System User's Guide \c
.qu
for the correct syntax for the 
command.  Specify only as many arguments as the command syntax allows
and reenter the command line.  Use a second command line for the remaining 
arguments, if appropriate. 
.sp 2
.ti -15
User # range is [0-15]
.sp
The user number specified in the command line is not supported
by the BIOS.  The valid range is enclosed in the square brackets in 
the error message.  Specify a user number between 0 and 15 (decimal) when you 
reenter the command line.
.in 0
.ll 65
.sp 2
.tc         E.5.2  CCP Internal Logic Error Messages
.sh
E.5.2  CCP Internal Logic Error Messages
.pp
The following message indicates an undefined failure of the BDOS Program Load 
Function (59).
.sp
Program Load Error
.sp
If you receive this message, contact the place you purchased your 
system for assistance.  
You should provide the information below.
.sp 2
.in 8
.ti -3
1) Indicate which version of the operating system you are using.
.sp
.ti -3
2) Describe your system's hardware configuration.
.sp
.ti -3
3) Provide sufficient information to reproduce the error.  Indicate 
which program was running at the time the error occurred.  If possible, 
you should also provide a disk with a copy of the program.
.in 0
.bp
.tc    E.6  DDT-8000 Error Messages
.he CP/M-8000 Programmer's Guide           E.6  DDT-8000 Error Messages
.sp
.sh
E.6  DDT-8000 Error Messages
.ix DDT-8000 error messages
.ix error message, DDT-8000
.pp 5
The CP/M-8000 debugger, DDT-8000, returns two types of error messages:  
nonfatal diagnostic error messages and fatal errors in the internal logic
of DDT-8000.
.sp 2
.tc         E.6.1  Diagnostic Error Messages
.sh
E.6.1  Diagnostic Error Messages 
.pp
Diagnostic error messages are returned at the console as the error occurs.  
The DDT-8000 error messages are listed below in alphabetic order
with explanations and suggested user responses.
.sp 2
.ce
.sh
Table E-7.  DDT-8000 Diagnostic Error Messages
.nf
.ll 60
.sp
.in 5
Message        Meaning
.fi
.sp
.in 20
.ti -15
Bad or non-existent RAM at HEX no.
.sp
This error occurs in response to a Set (S), Set Word (SW), 
or Set Longword (SL) command.  The message indicates one of two 
things.  
.in 23
.ti -3
.sp
1) The memory location at "HEX no." is  read-only, an I/O port, 
or nonexistent.  Use another location.
.ti -3
.sp
2) The memory location is damaged.  Check the hardware.
.sp 2
.in 20
.ti -15
Bad relocation bits
.sp
This message is returned from the BDOS Program Load Function
(59), and means one of two things.  
.in 23
.ti -3
.sp 
1) The command file has been corrupted.  Rebuild the file.  Reassemble or 
recompile the source file, and relink the file before you reenter the 
DDT-8000 command line.
.in 0
.ll 65
.bp
.ce
.sh
Table E-7.  (continued)
.nf
.ll 60
.sp
.in 5
Message        Meaning
.fi
.sp
.in 23
.ti -3
2) The file is linked to an absolute location in memory that is already 
occupied by DDT-8000.  Link the file to another location:  DDT-8000 occupies 
approximately 20K
of memory, and resides at the highest addresses within the TPA.  The 
recommended location for linking your file is the base address of
the TPA + 100H.  BDOS Function 63, Get/Set TPA Limits, returns 
the high and low boundaries of the TPA.
.sp 2
.in 20
.ti -15
Cannot create file
.sp
This error occurs during a Write (W) command.  The disk to 
which DDT-8000 is writing has no more directory space available: in effect, 
the disk is full.  If you have another drive available, reenter the Write 
(W) command and direct the file to the disk 
on that drive.  If you do not have another drive available, you 
must exit DDT-8000 (and lose the contents of memory).  Erase unnecessary 
files, if any, or insert a new disk.    
.sp 2
.ti -15
Cannot open file
.sp
This error occurs during a Read (R) command.  It indicates an 
incorrect user number, drive code, or filename.  Check the user number, drive 
code, and filename before you reenter the command line.
.sp 2
.ti -15
Cannot open program file
.sp
This message occurs in response to a Load for Execution (E) 
command.  It indicates 
an incorrect user number, drive code, or filename.  Check the user number, 
drive code, and filename before 
you reenter the command line.
.in 0
.ll 65
.bp
.ce
.sh
Table E-7.  (continued)
.nf
.ll 60
.sp
.in 5
Message        Meaning
.fi
.sp
.in 20
.ti -15
ERROR, no program or file loaded.
.sp
This error message occurs in response to a Value (V) 
command when you specify the command but no file is loaded.  Load a file 
before you reenter the V command.  The file can
be loaded with a Load for Execution (E) or Read (R) command, or by 
specifying  the filename when you invoke DDT-8000.
.sp 2
.ti -15
File too big -- read truncated
.sp
This message occurs during a Read (R) command when the 
file being read is too large to fit in memory.  DDT-8000 reads only 
the portion of the file
that can be read into the existing memory.  To debug this program, additional 
memory boards must be added to the system configuration.  
.sp 2 
.ti -15
File write error
.sp  
The disk to which DDT-8000 is writing is full or the 
disk contains a bad sector.  Retry the command.  If the error persists, and 
you have another
disk drive available, redirect the output to the disk on that 
drive.  If you 
do not have another drive available, you must exit DDT-8000.  
Use the STAT command to check the space on the disk.  If it is full, 
erase unnecessary files, if any, or insert a new disk.  Because the contents 
of memory 
are lost when you exit DDT-8000, you must reload the file 
in memory.  If the disk was not full, it has a bad sector.  You
should replace the disk.
.in 0
.ll 65
.bp
.ce
.sh
Table E-7.  (continued)
.nf
.ll 60
.sp
.in 5
Message        Meaning
.fi
.sp
.in 20
.ti -15
**illegal size field
.sp
This error occurs during a List (L) command.  The size field of the 
instruction being disassembled has an illegal value.  Use a Display (D) 
command to display the location of the error.  This error could be 
caused by one of three things:
.sp 
.in 23
.ti -3
1) The memory location being disassembled does not contain an 
instruction.  Ensure that the area selected is an instruction.  If not, 
reenter the L
command with a correct location.
.sp
.ti -3
2) The size field of the instruction has been corrupted.  Use the debugging 
commands in DDT-8000 to look for an error that causes the
program to overwrite itself.  Refer to the section in this manual 
on DDT-8000 for a complete description of the DDT-8000 commands and 
options.
.sp
.ti -3
3) An invalid instruction was generated by the compiler or assembler used to
create the program.  Recompile or reassemble the source file before you 
reinvoke DDT-8000.
.sp 2
.in 20
.ti -15
Insufficient memory or bad file header
.sp
This message occurs in response to a Load for Execution (E) command.  The error
could be caused by one of three things:
.in 0
.ll 65
.bp
.ce
.sh
Table E-7.  (continued)
.nf
.ll 60
.sp
.in 5
Message        Meaning
.fi
.sp
.in 20
.in 23
.sp
.ti -3
1) The system you are using does not have enough memory available.  Ensure 
that the program and DDT-8000 fit into the TPA. 
Exit DDT-8000.  Use the SIZE68 Utility to display the 
amount of space your program occupies in memory.  DDT-8000 is 
approximately 20K bytes.  The BDOS Get/Set TPA 
Limits Function (63) returns the 
high and low boundaries of the TPA.  If you do not have sufficient 
space in the TPA to execute your 
command file and DDT-8000 simultaneously, additional memory boards 
must be added to the system configuration.
.sp
.ti -3
2) The file is not a command file or has a corrupted header.  If the command 
file does not run, but
you are sure that your memory space is adequate, use the R command to look at
the file and check the format.  You may be trying to debug a file that is not
a command file.  If it is a command file, the header may have been corrupted.  
Reassemble or recompile the source file before you reenter the E command line.
If the error persists, it may be caused by an error in the internal 
logic of DDT-8000.  Contact the place you purchased your system for 
assistance.  
You should provide the information below.
.sp
.in 26
.ti -3
a. Indicate which version of the operating system you are using.
.sp
.ti -3
b. Describe your system's hardware configuration.
.sp
.ti -3
c. Provide sufficient information to reproduce the error.  Indicate 
which program was running at the time the error occurred.  If possible, 
you should also provide a disk with a copy of the program.
.in 0
.ll 65
.bp
.ce
.sh
Table E-7.  (continued)
.nf
.ll 60
.sp
.in 5
Message        Meaning
.fi
.sp
.in 23
.sp
.ti -3
3) The command file you are debugging is linked to an absolute location in 
memory that is already occupied by DDT-8000.  DDT-8000 is approximately 20K 
bytes, and usually resides in the highest 
addresses of the TPA.  The recommended location for linking your file is the 
base address of the TPA + 100H.  The BDOS Get/Set TPA 
Limits Function (63) returns the high and low boundaries of the TPA.
.in 20
.sp 2
.ti -15
Read error
.sp
This message may indicate one of three things.  Try the operation again.  If 
the error persists, try the responses indicated:
.sp
.in 23
.ti -3
1) A write error at the time the file was created.  You must recreate the 
file.  If the error reoccurs, or if 
you cannot write to the disk, the disk is bad.
.sp
.in 23
.ti -3
2) A bad disk.  Use PIP or COPY to copy the file from the bad
disk to a new disk.  Any files that cannot be copied must be 
recreated or replaced from backup files.  Discard the damaged 
disk.
.sp
.ti -3
3) A hardware error.  If the error persists, check your hardware.
.in 20
.sp 2
.ti -15
unknown opcode
.sp
This error occurs in response to a List (L) command if 
the memory location being disassembled does not contain a valid instruction.
The error may have been caused by one of three things:  
.sp
.in 0
.ll 65
.bp
.ce
.sh
Table E-7.  (continued)
.nf
.ll 60
.sp
.in 5
Message        Meaning
.fi
.sp
.in 23
.ti -3
1) You gave the L command the wrong address.  Reenter the L command with the 
correct address.
.sp
.ti -3
2) The file is not a command file.  Ensure that the file you specify is a 
command file and reenter the L command.
.sp
.ti -3
3) The command file has been corrupted.  Reassemble or recompile the source 
file 
before you reread it into memory with a Load for Execution (E) or 
Read (R) command, as appropriate.  If the problem persists, use the 
debugging commands in DDT-8000 to look for an error in the program 
that causes it to overwrite itself.  Refer to the section in this manual 
on DDT-8000 for a complete description of the DDT-8000 commands and 
options.
.in 0
.ll 65
.sp 2
.tc         E.6.2  DDT-8000 Internal Logic Error Messages
.sh
E.6.2  DDT-8000 Internal Logic Error Messages
.pp
This section lists fatal errors in the internal logic of DDT-8000.  If you
receive one of these messages, contact the place you purchased 
your system for assistance.  
You should provide the information below.
.sp 2
.in 8
.ti -3
1) Indicate which version of the operating system you are using.
.sp
.ti -3
2) Describe your system's hardware configuration.
.sp
.ti -3
3) Provide sufficient information to reproduce the error.  Indicate 
which program was running at the time the error occurred.  If possible, 
you should also provide a disk with a copy of the program.
.in 0
.sp 2
Errors:
.sp
.in 3
.nf
o illegal instruction format #
.sp
o Unknown program load error
.fi
.in 0
.bp
.tc    E.7  DUMP Error Messages
.he CP/M-8000 Programmer's Guide              E.7  DUMP Error Messages
.sh
E.7  DUMP Error Messages
.pp 5
.ix DUMP error messages
.ix error message, DUMP
DUMP returns fatal, diagnostic error messages at the 
console.  The DUMP error messages are listed below in alphabetic order with
explanations and suggested user responses.
.sp 2
.ce
.sh
Table E-8.  DUMP Error Messages
.ll 60
.in 5
.sp
.nf
Message        Meaning
.fi
.sp
.in 20
.ti -15
Unable to open filename 
.sp
Either the drive code for the input file indicated by the variable
"filename" is incorrect, or the filename is misspelled.  Check the filename 
and drive code before you reenter the DUMP command line.
.sp 2
.ti -15
Usage:  dump [-shhhhhh] file
.sp
The command line syntax is incorrect.  The correct syntax is given in the 
error message.  Specify the DUMP command and the filename.  If you want to 
display the contents of the file from a specific address in the 
file, specify the -S option followed by the address.  Refer to 
the section in this manual on the DUMP Utility for a
discussion of the DUMP command line and options.
.in 0
.ll 65
.sp 2
.tc    E.8  LO68 Error Messages
.he CP/M-8000 Programmer's Guide              E.8  LO68 Error Messages
.sh
E.8  LO68 Error Messages
.pp 5
.ix LO68 error messages
.ix error messages, LO68
The CP/M-8000 Linker, LO68, returns two types of fatal error messages: 
diagnostic and logic.  Both types of fatal error 
messages have the following format:
.sp
.ti 8
:  error message text
.sp
The colon (:) indicates that the error message comes from LO68.  The "error 
message
text" describes the error.  
.nx appe3
