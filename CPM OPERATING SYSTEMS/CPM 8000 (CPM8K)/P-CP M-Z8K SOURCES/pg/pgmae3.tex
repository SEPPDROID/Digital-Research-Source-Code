.cs 5
.mt 5
.mb 6
.pl 66
.ll 65
.po 10
.hm 2
.fm 2
.he CP/M-8000 Programmer's Guide              E.8  LO68 Error Messages
.ft All Information Presented Here is Proprietary to Digital Research
.sp 2
.tc         E.8.1  Fatal Diagnostic Error Messages
.sh
E.8.1  Fatal Diagnostic Error Messages
.pp
A fatal diagnostic error prevents your program from linking.  When the error 
is caused by a full disk, erase the partial file that LO68 created on the 
disk that received the error to ensure that you do not use the file.  
The LO68 diagnostic errors are listed below in alphabetic order 
with explanations and suggested user responses.
.bp
.ce
.sh
Table E-9.  LO68 Fatal Diagnostic Error Messages
.ll 60
.in 5
.sp
.nf
Message        Meaning
.fi
.sp
.in 20
.ti -15
: duplicate definition in  p,filename
.sp
The symbol indicated by the variable "p" is defined twice.  
The variable "filename" indicates 
the file in which the second definition occurred.  Rewrite the source code.  
Provide a unique definition for each 
symbol and reassemble or recompile the source code before you relink the file.
.sp 2
.ti -15
: file format error:  filename
.sp
The file indicated by the variable "filename" is either not an 
object file or the file has been corrupted.  Ensure that the file is an 
object file, output by the assembler or 
compiler.  Reassemble or recompile the file before you relink it.
.sp 2
.ti -15
: File Format Error:  Invalid symbol flags = flags
.sp
LO68 does not recognize the symbol flags indicated by the variable
"flags."  The file LO68 read is either not an object file or it has been 
corrupted.  Ensure that the file is an object file, output by the assembler or 
compiler.  Reassemble or recompile the file before you relink it.
.sp 2
.ti -15
: File Format Error:  invalid relocation flag in filename
.sp
The contents of the file indicated by the variable 
"filename" are incorrectly formatted.  The file either is not an object file 
or has been corrupted.  Ensure that the file is an object file, output by the 
assembler or 
compiler.  If the file is an object file but this error occurs, the file
has been corrupted.  Reassemble or recompile the file before you relink it.
.in 0
.ll 65
.bp
.ce
.sh
Table E-9.  (continued)
.ll 60
.in 5
.sp
.nf
Message        Meaning
.fi
.sp
.in 20
.ti -15
: File Format Error:  no relocation bits in filename
.sp
The file indicated by the variable "filename" either is not an 
object file or has been corrupted.  Ensure that the file is an object file, 
output by the assembler or 
compiler.  If the file is an object file but this error occurs, then the file
has been corrupted.  Reassemble or recompile the file before you relink it.
.sp 2
.ti -15
: Illegal option  p
.sp
The option in the command line indicated by the variable "p" is 
invalid.  Supply a valid option and relink.
.sp 2
.ti -15
: Invalid lo68 argument list
.sp
This message indicates format errors or invalid options in the 
command line.  Examine the command line to locate the error.  Correct the 
error and relink.
.sp 2
.ti -15
: output file write error
.sp
The disk to which LO68 is writing is full.  Erase unnecessary files, if any, 
or insert a new disk before you reenter the LO68 command line.
.sp 2
.ti -15
: read error on file:  filename
.sp
The object file indicated by the variable "filename," does not 
have enough bytes.  
The file either is incorrectly formatted or has been corrupted.  This error 
is commonly caused when the input to LO68 is a partially
assembled or compiled object file.  The assembler, AS68, and some 
compilers create partial object files 
when they receive the "disk full abort" message while assembling or 
compiling a file.  Ensure that the file is a complete object file.  
Reassemble or recompile the file before you relink it.  
.in 0
.ll 65
.bp
.ce
.sh
Table E-9.  (continued)
.ll 60
.in 5
.sp
.nf
Message        Meaning
.fi
.sp
.in 20
.ti -15
: symbol table overflow
.sp
The object code contains too many symbols for the size of the 
symbol table.  Rewrite the source code to use fewer symbols.  Reassemble or 
recompile the source code before you relink the file.
.sp 2
.ti -15
: Unable to create  filename
.sp
Either the output file indicated by "filename" has an invalid drive 
code, or the disk to which LO68 is writing is full.  Check the drive code.  
If it is correct, 
the disk is full.  Erase unnecessary files, if any, or insert a new disk
before you reenter the LO68 command line.
.sp 2
.ti -15
: unable to open  filename
.sp
The filename indicated by the variable "filename" is invalid, 
or the file does not exist.  Check the filename before you reenter the LO68 
command line.
.sp 2
.ti -15
: Unable to open temporary file:  filename
.sp
Either the file, indicated by "filename", has an invalid drive 
code, specified by the "f" option, or the disk to which LO68 is writing 
is full.  Check the drive code. If it is correct, the disk is full.  Erase 
unnecessary files, if any, or insert a new disk before you 
reenter the LO68 command line.
.sp 2
.ti -15
: Undefined symbol(s)
.sp
The symbol or symbols which are listed one per line on the lines 
following the error message are undefined.  Provide a valid definition and 
reassemble the source code before you reenter the LO68 command line.
.in 0
.ll 65
.bp
.tc         E.8.2  LO68 Internal Logic Error Messages
.sh
E.8.2  LO68 Internal Logic Error Messages
.pp
This section lists messages indicating fatal errors in the internal 
logic of LO68.  If you receive one of these messages, contact the place you 
purchased your system for assistance.  You should provide the information 
below.
.sp
.in 8
.ti -3
1) Indicate which version of the operating system you are using.
.sp
.ti -3
2) Describe your system's hardware configuration.
.sp
.ti -3
3) Provide sufficient information to reproduce the error.  Indicate 
which program was running at the time the error occurred.  If possible, 
you should also provide a disk with a copy of the program.
.in 0
.sp 2
Errors:
.sp
.nf
.in 3
o : asgnext botch

o : finalwr: text size error

o : relative address overflow at lx in sn

o : seek error on file  filename

o : short address overflow in  filename

o : unable to reopen filename
.fi
.in 0
.sp 2
.tc    E.9  NM68 Error Messages
.he CP/M-8000 Programmer's Guide              E.9  NM68 Error Messages
.sh
E.9  NM68 Error Messages
.pp 5
.ix NM68 error messages
.ix error messages, NM68
NM68 returns fatal diagnostic error messages at the console.  The NM68 error
messages are listed below in alphabetic order with explanations and suggested 
user responses.
.bp
.ce
.sh
Table E-10.  NM68 Error Messages
.ll 60
.in 5
.sp
.nf
Message        Meaning
.fi
.sp
.in 20
.ti -15
file format error:  filename
.sp
The input file indicated by the variable "filename" is neither an
object file nor a command file.  This message can also indicate a corrupted
file.  NM68  prints the symbol table of an object file or a command 
file.  Ensure that the file is one of these types of file.  If the 
file is an object or 
command file and you receive this message, the file is corrupted.  
Rebuild the file with the compiler or assembler.  If the file is a 
command file,  relink it.  Reenter the NM68 command line. 
.sp 2
.ti -15
read error on file:  filename
.sp
The input file indicated by the variable "filename" is truncated.  Rebuild 
the file with the compiler or assembler.  If the file is a 
command file,  relink it.  Reenter the NM68 command line. 
.sp 2
.ti -15
unable to open  filename
.sp
The filename indicated by the variable "filename" is incorrect.  Check the 
spelling of the filename and reenter the command line.
.sp 2
.ti -15
Usage:  nm68 objectfile
.sp
The command line syntax is incorrect.  Use the syntax given in the error 
message and reenter the command line.
.in 0
.ll 65
.sp 2
.tc    E.10 RELOC Error Messages
.he CP/M-8000 Programmer's Guide            E.10  RELOC Error Messages
.sh
E.10  RELOC Error Messages
.pp 5
.ix RELOC error messages
.ix error message, RELOC
The Relocation Utility (RELOC) returns fatal error messages at the 
console.  RELOC error messages are listed below in alphabetic order with
explanations and suggested user responses.
.bp
.ce
.sh
Table E-11.  RELOC Error Messages
.ll 60
.sp
.nf
.in 5
Message        Meaning
.fi
.sp
.in 20
.ti -15
create filename
.sp
Either the drive code for the output file indicated by the 
variable "filename" is incorrect, or the disk to which RELOC is writing
is full.  Check the drive code.  If it is correct, the disk is full.  Erase
unnecessary files, if any, or insert a new disk before you reenter the RELOC
command line.
.sp 2
.ti -15
Cannot open filename
.sp
The input file indicated by the variable "filename" does not 
exist.  Ensure that you type the correct filename when you reenter the 
RELOC command line.
.sp 2
.ti -15
Cannot re-open  filename
.sp
This error message indicates a hardware error.  Check the hardware for 
errors.  This error most often occurs in the disk, disk drive, or memory.
.sp 2
.ti -15
File format error:  filename
.sp
This error occurs because the first word in the header record of the 
command file must contain the value 601AH and the file must contain 
relocation bits.  If your file does not meet these criteria, you cannot use 
RELOC.
.sp
.in 23
.ti -3
1) The file indicated by the variable "filename" is not a command file with
contiguous program segments (the first word in the header record is 
601AH).  If the file is an object file, link it before you reenter the RELOC
command line.
.sp
.ti -3
2) The file does not have relocation bits because it is already linked 
to an absolute location.  Use the original source file that contains 
relocation bits with RELOC.
.in 0
.ll 65
.bp
.ce
.sh
Table E-11.  (continued)
.ll 60
.sp
.nf
.in 5
Message        Meaning
.fi
.sp
.in 20
.ti -15
Illegal base address=hex no.
.sp
The odd base address indicated by the variable "hex no." is invalid 
under CP/M-8000.  Base addresses must be even.  Specify an even base address 
and reenter the RELOC command line.
.sp 2
.ti -15
Illegal option:  x
.sp
The option specified for the RELOC command must be -b.  The invalid
option is indicated by the variable "x".  Replace the invalid option with -b 
and reenter the RELOC command line.
.sp 2
.ti -15
Illegal reloc = x at address
.sp
This message may indicate one of two things:
.in 23
.sp
.ti -3
1) The command file is truncated or corrupted.  RELOC recognized the
error because the relocation value indicated by the variable "x" is invalid.  
The variable "address" indicates the location in memory of the invalid 
relocation value.  Rebuild the file.  Reassemble, or recompile, and relink 
the file before you reenter the RELOC command line.
.sp
.ti -3
2) The file has no relocation bits.  Use the original source code with 
relocation bits and try again.
.in 20
.sp 2
.ti -15
Read error on filename
.sp
The input file indicated by the variable "filename" is truncated or 
corrupted.  Rebuild the file.  Reassemble, or recompile, and relink 
the file before you reenter the RELOC command line.
.in 0
.ll 65
.bp
.ce
.sh
Table E-11.  (continued)
.ll 60
.sp
.nf
.in 5
Message        Meaning
.fi
.sp
.in 20
.ti -15
16-bit overflow at address
.sp
The address indicated by the variable "address" cannot contain a
16-bit quantity.  Source code that uses 16-bit offsets must fit in the first 
64K bytes of memory.  BDOS Function 63, Get/Set TPA Limits, returns the high
and low boundaries of the memory available for loading programs.  SIZE68
displays the amount of memory space a program occupies.  Use the 
Get/Set TPA Limits Function and SIZE68 to ensure that the program 
fits in the first 64K of memory.  If the program does not fit, you 
must rewrite the source code to use 32-bit offsets. 
.sp 2
.ti -15
.nf
Usage: reloc -bhhhhhh input output
.ti -15
       where:  hhhhhh is new base address
.ti -15
               input is relocatable file
.ti -15
               output is absolute file
.fi
.sp
This message indicates a syntax error in the RELOC command line.
The correct syntax is given in the error message.  Retype the command line 
with the correct syntax.  Refer to the section in this manual on
the RELOC Utility for more detailed information on the command line syntax.
.sp 2
.ti -15
Write error on filename   Offset = x  data = x  error = x
.sp
The disk to which RELOC is writing is full.  Erase unnecessary files, if any, 
or insert a new disk before you reenter the RELOC command line.	
.in 0
.ll 65
.sp 2
.tc    E.11 SENDC68 Error Messages
.he CP/M-8000 Programmer's Guide          E.11  SENDC68 Error Messages
.sh
E.11  SENDC68 Error Messages
.pp 5
.ix SENDC68 error messages
.ix error messages, SENDC68
SENDC68 returns two types of fatal error messages:  diagnostic and internal 
logic error messages.  
.bp
.tc         E.11.1  Diagnostic Error Messages
.sh
E.11.1  Diagnostic Error Messages
.pp
The SENDC68 diagnostic error messages are listed below in alphabetic order 
with explanations and suggested user responses.
.sp 2
.ce
.sh
Table E-12.  SENDC68 Diagnostic Error Messages
.ll 60
.in 5
.nf
.sp
Message        Meaning
.fi
.sp
.in 20
.ti -15
file format error:  filename
.sp
The file indicated by the variable "filename" is not a command file.
The file input to SENDC68 must be a command file, output by the linker 
(LO68).  Ensure that the file specified is a command file.
.sp 2
.ti -15
read error on file:  filename
.sp
The file indicated by the variable "filename" is truncated.  Rebuild the file 
by recompiling or reassembling, and relink it before you reenter the SENDC68 
command line.
.sp 2
.ti -15
unable to create filename
.sp
This message indicates an invalid drive code for the 
output file indicated by the variable "filename".  It can also mean that the
disk to which SENDC68 is writing is full.  Check the drive code.  If it is 
correct, the disk is full.
Erase unnecessary files, if any, or insert a new disk before you reenter the
SENDC68 command line.
.sp 2
.ti -15
unable to open filename
.sp
The input file indicated by the variable "filename" 
does not exist.  Check the filename and retype the SENDC68 command line.
.sp 2
.mb 5
.fm 1
.ti -15
Usage:  sendc68  [-] commandfile [outputfile]
.sp
This message indicates a syntax error in the SENDC68 command line.
The correct syntax is given in the error message.  Retype the command line 
using the correct syntax.
.in 0
.ll 65
.bp
.mb 6
.fm 2
.tc         E.11.2  SENDC68 Internal Logic Error Messages
.sh
E.11.2  SENDC68 Internal Logic Error Messages
.pp
The following is a fatal error in the internal logic of SENDC68.  
.sp
.ti 8
INTERNAL LOGIC ERROR:  seek error on file  filename
.sp
If you receive this message, contact the place you purchased your 
system for assistance.  
You should provide the information below.
.sp 2
.in 8
.ti -3
1) Indicate which version of the operating system you are using.
.sp
.ti -3
2) Describe your system's hardware configuration.
.sp
.ti -3
3) Provide sufficient information to reproduce the error.  Indicate 
which program was running at the time the error occurred.  If possible, 
you should also provide a disk with a copy of the program.
.in 0
.sp 2
.tc    E.12 SIZE68 Error Messages
.he CP/M-8000 Programmer's Guide           E.12  SIZE68 Error Messages
.sh
E.12  SIZE68 Error Messages
.pp 5
.ix SIZE68 error messages
.ix error messages, SIZE 68
SIZE68 returns fatal, diagnostic error messages at the console.  The SIZE68
error messages are listed below in alphabetic order with explanations and
suggested user responses.
.sp 2
.ce
.sh
Table E-13.  SIZE68 Error Messages
.ll 60
.in 5
.nf
.sp
Message        Meaning
.fi
.sp
.in 20
.ti -15
File format error:  filename
.sp
The file indicated by the variable "filename" is neither an 
object file nor a command file.  SIZE68 requires either an object file, 
output by the assembler or
the compiler, or a command file, output by the linker.  Ensure that the
file specified is one of these and reenter the SIZE68 command line.
.sp 2
.ti -15
read error on  filename
.sp
The file indicated by the variable "filename" is truncated.  Rebuild the 
file.  Reassemble or recompile, and relink the 
source file before you reenter the SIZE68 command line.
.in 0
.ll 65
.bp
.ce
.sh
Table E-13.  (continued)
.ll 60
.in 5
.nf
.sp
Message        Meaning
.fi
.sp
.in 20
.ti -15
unable to open  filename
.sp
Either the drive code is incorrect, or the file indicated by the 
variable "filename" does not exist.  Check the drive code and filename.  
Reenter the SIZE68 command line.
.in 0
.ll 65
.sp 2
.ce
End of Appendix E
.nx appf
